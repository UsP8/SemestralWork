%% LyX 2.0.1 created this file.  For more info, see http://www.lyx.org/.
%% Do not edit unless you really know what you are doing.
\documentclass[english]{article}
\usepackage[T1]{fontenc}
\usepackage[latin9]{inputenc}
\usepackage{babel}
\begin{document}

\title{Project for Markup language}


\author{Student: Adam Vod�k (InfoA)}


\date{04.03.2017}

\maketitle

\part{Table of contents:}

\tableofcontents{}


\part{\newpage{}Introduction:}


\section{Why should we use XML language?}

Although XML was created to overcome the shortcomings of HTML, it
becomes more and more popular all over the world. It is even considered
a critical tech- nology which can change the future of the web. And
there are some reasons for this: Through XML, it is possible to exchange
documents and data of a variety of formats easily and perfectly. Because
XML can process and present data in various ways, it is possible for
each user to exchange his data/documents in dif- ferent formats. {[}8{]}

Users can easily handle structured databases on the web. XML can be
easily changed into database les and vice-versa. So, users can directly
import a variety of data on the web to their own database. Better
Internet searching possible. Because XML eliminates the uncertainties
of information, the speed and accuracy of Internet searching will
be improved. Authors and providers can design their own document types.
Document types can be explic- itly tailored to users, so the cumbersome
fudging that has to take place with HTML to achieve special eects
can become a thing of the past. Information content can be richer
and easier to use because the hypertext linking abilities of XML are
much greater than those of HTML.{[}6{]}


\section{What actualy XML is for?}

Really Simple Syndication, arguably the most popular content syndication
format on the web today, is simply an XML-based schema. XHTML is essentially
the HTML markup language you're well familiar with only adjusted to
conform to XML structure, syntax, and validation. So it is yet another
XML-based language. WML is the XML-based markup for WAP services.
The list goes on. By being built upon XML, these case-specic schemas
have inherited XML's rich infrastructure for free. No complaints here!
Outside of the many structured language implementations of XML, the
most obvious role of XML as a stand-alone \textquotedbl{}le format\textquotedbl{}
is to represent data apart from visual markup. But why would you want
to store content in XML as opposed to a database? To one point of
view, XML is a more neutral and ac- cessible container for your data
to reside within one that does not require as much database adapters
and local conguration. Although it's also common to nd XML used as
streamed data from a connected application that is, as an environment-neutral
communications vehicle XML content is designed to be equally efective
stored. {[}1{]}

The AJAX development model is based on sending XML data back and forth
between browser and server, allowing JavaScript to use the received
data to update a web page without refreshing. Flash-based Rich Internet
Applications can, of course, directly access XML les and streams by
way of the XML class and connector component in Flash 8. In ActionScript
3, XML is now treated as a native data type. As opposed to opening
and reading an XML le, and then parsing and acting upon it, Ac- tionScript
3 allows you simply to refer to an XML file as a variable, and manipulate
your data directly from that point on. You can expect to see more
and more coding environments and frameworks move this way as XML entrenches
itself further as data's lingua franca of the web. XML is very eective
at providing aggregate views of content, such as syndicating a range
of documents. Really Simple Syndication, is one format widely used
for syndication, and in most cases is autogenerated by the system
managing the site content. Web log frameworks like Movable Type and
Blogger generate and update a static Really Simple Syndication le
on your server whenever a new post is published to the system, which
Really Simple Syndication savvy clients can then read and use (in
most cases an internal stylesheet) to create the visual layout of
the structured content data in the Really Simple Syndication /XML
feed.{[}5{]}


\part{\newpage{}Methodology}


\section{Methodology}

Due to the growing use of XML data format in global information, an
eective XML data management system is needed. An Enabled XML DB is
one of 2 the recent widely accepted approaches to store XML documents.
This ability coupled with the increase use of XML data in dierent
areas have triggered the need for a better method to structure a large
data in order to improve query performance. Issues concerning the
ways to eciently partition large XML documents into a more manageable
form are yet to be addressed. At the same time, it is essential to
ensure that the partitioning method maintains the preservation of
XML data hierarchical structure. For this reason, this paper introduces
OXDP that structures large XML data logically by partitioning them
into object based XML components. An evaluation is shown to demonstrate
the efectiveness of OXDP in XML partitioning which subsequently has
the potential of improving query performance in Enabled XML DB environments.


\section{Used technologies}

In my project I have been really long time looking and searching on
the Internet what have I actualy used because my friend programmator
was helping me with this project. But after couple of minutes I have
found that actualy I have used DTD (Document Type Denition) A standard
for dening the legal elements and attributes in an XML document and
XSL stylesheet transformation to HTML. Unluckily I wasn't able to
transform my project into HTML table and put it on the web.{[}3{]}


\part{\newpage{}Elements description}
\begin{verse}
What is element from the global POV?\end{verse}
\begin{itemize}
\item An element describes the data that it contains. 
\item Elements can also contain other elements and attributes. 
\item When an element denition contains additional elements or attributes,
it is a complex type
\end{itemize}

\section{Tags and attributes}

The Extensible Stylesheet Language Transformations (XSLT) APIs can
be used for many purposes. For example, with a suciently intelligent
stylesheet, you 3 could generate PDF or PostScript output from the
XML data. But generally, XSLT is used to generate formatted HTML output,
or to create an alternative XML representation of the data.In this
section, you'll use an XSLT transform to translate XML input data
to HTML output.

An attribute is a named simple-type denition that cannot contain other
ele- ments. Attributes can also be assigned an optional default value
and they must appear at the bottom of complex-type denitions. Additionally,
if multiple attributes are declared, they may occur in any order.
{[}2{]}


\section{Describe the implementation regarding the real usage}

Elsevier Science, a publisher of scientic, technical, and medical
information, uses Mark Logic's Content Interaction Server to manage
more than two ter- abytes of data: ve million full-text journal articles,
60 million citations and abstracts, thousands of complete books, and
ve thousand informational pam- phlets. The system is used to search
and transform documents. Raining Data's TigerLogic XML Data Management
Server is used in similar fashion by large scientic publishing companies.{[}9{]}


\part{\newpage{}Transformation description}
\begin{verse}
The Extensible Stylesheet Language Transformations (XSLT) APIs can
be used for many purposes. 

For example, with a suciently intelligent stylesheet, you 3 could
generate PDF or PostScript output from the XML data. 

But generally, XSLT is used to generate formatted HTML output, or
to create an alternative XML representation of the data.In this section,
you'll use an XSLT transform to translate XML input data to HTML output.{[}4{]}
\end{verse}

\part{My conclusion}

User is able to easily handle structured databases on the web. XML
can be very simply changed into database files and vice-versa. So,
users can directly import a variety of data on the web to their own
database. However, it is a technology that has now captured desktops,
markets and mind share.\pagebreak{}

There is no doubt that XML is prety usefull tool and I have started
using it since this semestre and I am pretty glad I do!


\part{References:}
\begin{itemize}
\item http://www.adobe.com/cn/devnet/dreamweaver/articles/spry_creating_xml_data_set.html {[}1{]}
\item https://docs.oracle.com/javase/tutorial/jaxp/xslt/transformingXML.html
{[}2{]}
\item http://webdesign.about.com/od/xmlglossary/tp/xml-terms.htm {[}3{]}
\item https://techterms.com/definition/xml {[}4{]}
\item https://www.codeproject.com/Articles/20486/What-makes-XML-such-a-great-technology {[}5{]}
\item http://www.rpbourret.com/xml/UseCases.htm
{[}6{]}
\item https://www.w3schools.com/xml {[}7{]}
\item https://www.w3schools.com/xml/xsl_languages.asp\end{itemize} {[}8{]}
\end{document}
